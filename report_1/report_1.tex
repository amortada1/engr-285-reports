\documentclass[12pt]{iopart} % Document class declaration

% package "imports"
\usepackage{graphicx}
\usepackage{IEEEtrantools}

% Custom macros
\gdef\mcm{r@{.}l@{ ± }r@{.}l} % Multi Column Measurement; Used for decimal aligning & ± aligning
\gdef\mch#1{\multicolumn{4}{l}{#1}} % Multi Column Header; Used for decimal aligning & ± aligning
\gdef\mcmnd{r@{ ± }l} % Multi Column Measurement No Decimal; Used for ± aligning when the values don't need a decimal point
\gdef\mchnd#1{\multicolumn{2}{l}{#1}} % Multi Column Header No Decimal; Used for  ± aligning when the values don't need a decimal point
\gdef\sci#1#2{#1 \times 10^{#2}}
\gdef\units#1{~\mathrm{#1}}

%%%%%%%%%%%%%%%%%%%% Document Starts %%%%%%%%%%%%%%%%%%%%
\begin{document}

%%%%%%%%%%%%%%%%%%%% Title Page %%%%%%%%%%%%%%%%%%%%
\title{Title}
\author{Ali Mortada}
\vspace{10pt}
\begin{indented}
  \item[]Mt.~San Antonio College, ENGR 285, CRN 43464
  \item[]Date
\end{indented}
\newpage

%%%%%%%%%%%%%%%%%%%% Purpose %%%%%%%%%%%%%%%%%%%%
\section{Purpose and Hypothesis}

The purpose of the experiment is \emph{blah blah blah}.

%%%%%%%%%%%%%%%%%%%% Materials %%%%%%%%%%%%%%%%%%%%
\section{Materials}

\begin{itemize}
\item
  Thingamajig 1
\item
  Thingamajig 2
\end{itemize}

%%%%%%%%%%%%%%%%%%%% Procedures %%%%%%%%%%%%%%%%%%%%
\section{Procedures}

\begin{enumerate}
\def\labelenumi{\arabic{enumi}.}
\item
  Tingaling 1

  \begin{itemize}
    \item
      Mini Tingaling 1
    \item
      Mini Tingaling 2
  \end{itemize}
\item
  Tingaling 2
\end{enumerate}

%%%%%%%%%%%%%%%%%%%% Results %%%%%%%%%%%%%%%%%%%%
\section{Results}

Table \ref{tab:short_cable_properties} contains the measured properties of the short fiber optic cable.
$l$ is the length of that cable, and $m$ is the mass of that cable.

\begin{table}[htbp]
\caption{\label{tab:short_cable_properties}
Short Fiber Optic Cable Properties
}
\begin{indented}\lineup\item[]\begin{tabular}{@{}cr@{ ± }l}
\br
  Property & \multicolumn{2}{c}{Measurement} \\
\mr
  $l$      & 148.8 & 0.2 mm \\
  $m$      & 0.57 & 0.01 g  \\
\br
\end{tabular}\end{indented}\end{table}

Table \ref{tab:main_measurements} contains the main measurements made for the various colors of light and long fiber optic cables used.
$\lambda$ is the peak wavelength output by the LED used.
$M$ is the mass of the long fiber optic cable used.
$t$ is the travel time for the light pulse to pass through the long fiber optic cable.

\begin{table}[htbp]
\caption{\label{tab:main_measurements}
Main Measurements \\
Note: UTD means ``unable to detect''
}
\begin{indented}\lineup\item[]\begin{tabular}{@{}lll\mcm\mcm}
\br
  Trial & Color    & $\lambda$ (nm) & \mch{$M$ (g)}   & \mch{$t$ (ns)} \\
\mr
  1     & Red      & 645 & 75&10 & 0&01     & 99&85 & 1&53 \\
  2     & Green    & 522 & 75&10 & 0&01     & \multicolumn{4}{l}{UTD through short cable} \\
  3     & Blue     & 470 & 75&10 & 0&01     & \multicolumn{4}{l}{UTD through short cable} \\
  4     & Infrared & 940 & 75&10 & 0&01     & \multicolumn{4}{l}{UTD through long cable} \\
  5     & Infrared & 940 & 37&08 & 0&01     & \multicolumn{4}{l}{UTD through long cable} \\
  6     & Infrared & 940 &  3&75 & 0&01     & \multicolumn{4}{l}{UTD clearly through long cable} \\
  7     & Infrared & 940 &  1&87 & 0&01     & \multicolumn{4}{l}{UTD clearly through long cable} \\
\br
\end{tabular}\end{indented}\end{table}



%%%%%%%%%%%%%%%%%%%% Uncertainty and Equations %%%%%%%%%%%%%%%%%%%%
\section{Uncertainty and Equations}

The length of the short cable ($l$) was measured using a caliper.
It had a labeled uncertainty of 0.2 mm, so that was used as the uncertainty.
The masses ($m$ and $M$) were measured using an electronic balance that reported values to within $0.01 \units{g}$, so that was used as the uncertainty.

The material the fiber optic cable is made of has an index of refraction ($n$) of 1.49, according to the datasheet.
Thus, light slows down when traveling in the cable compared to the speed of light in a vacuum ($c$, accepted value of $\sci{3.00}{8} \units{m/s}$).
Equation \ref{eq:index_of_refraction} can be used to calculate the speed of light in the cable ($v$).
\begin{IEEEeqnarray}{rCl}
  v & = & \frac{c}{n} \label{eq:index_of_refraction}
\end{IEEEeqnarray}

The length of the long fiber optic cable ($L$) was determined indirectly based on its mass ($M$) and the linear density of that type of cable ($\mu$).
The linear density can be calculated using equation \ref{eq:linear_density}, which uses the more easily measured mass ($m$) and length ($l$) of the short fiber optic cable.
\begin{IEEEeqnarray}{rCl}
  \mu & = & \frac{m}{l} \label{eq:linear_density}
\end{IEEEeqnarray}
  From there, the length of the long cable $L$ can be calculated using equation \ref{eq:long_cable_length}.
\begin{IEEEeqnarray}{rCl}
  L & = & \frac{M}{\mu} \label{eq:long_cable_length} \\
    & = & \frac{Ml}{m} \nonumber
\end{IEEEeqnarray}

%%%%%%%%%%%%%%%%%%%% Conclusion %%%%%%%%%%%%%%%%%%%%
\section{Conclusion}

We conclude that roses are red and violets are blue.

%%%%%%%%%%%%%%%%%%%% Citations %%%%%%%%%%%%%%%%%%%%
\section{Citations}

\begin{thebibliography}{9}

\bibitem{reference_name}
  Mister Man,
  \textit{Title of Reference},
  Publisher,
  Year,
  Pages or Hyperlink.

\end{thebibliography}

\end{document}
%%%%%%%%%%%%%%%%%%%% Document Ends %%%%%%%%%%%%%%%%%%%%
